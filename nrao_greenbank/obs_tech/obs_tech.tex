% Cover letter using letter.sty
\documentclass[11pt]{letter} % Uses 10pt
%Use \documentstyle[newcent]{letter} for New Century Schoolbook postscript font
% the following commands control the margins:
\topmargin=-1.2in    % Make letterhead start about 1 inch from top of page 
\textheight=14in  % text height can be bigger for a longer letter
\oddsidemargin=0pt % leftmargin is 1 inch
\textwidth=6.5in   % textwidth of 6.5in leaves 1 inch for right margin

\begin{document}\pagenumbering{gobble}

\longindentation=0pt                       % needed to get closing flush left
\let\raggedleft\raggedright                % needed to get date flush left
 

\begin{flushleft}
{\large\bf Eamon O'Gorman \ \ \ \ \ \ \ \ \ \ \ \textit{Radio Astronomy Experience}}
\end{flushleft}
\medskip\hrule height 1pt
\begin{flushright}


\end{flushright} 
%\vfill % forces letterhead to top of page

 \begin{itemize}

 
    \item I have worked extensively with millimeter interferometric line emission data during my Ph.D. studies. These data were obtained with the CARMA heterogeneous interferometer at Ceder Flat, California, in three different array configurations. The raw data were flagged, calibrated, and imaged using CASA. The analysis required knowledge of interferometric spatial scales and imaging techniques such as CLEAN and multi-scale CLEAN. \\
    
    \item I have also worked with sub-millimeter line emission data from the SMA interferometer located atop Mauna Kea in Hawaii, and the NASA/DLR SOFIA telescope. \\ 
    
    \item I have planned and prepared two sets of observations of multi-frequency centimeter continuum observations with the Karl G. Jansky Very Large Array (PI: G. Harper, ID: 10C-105, and PI: E. O'Gorman, ID: 12A-472). Different frequencies required different observing strategies, such as choice of calibrator (flux and phase) and length of observing scans. I flagged, calibrated, and imaged all of these data in CASA. The large bandwidth now provided by the new VLA required additional steps to mitigate RFI, and to create wideband images.\\ 
        
    \item I am currently working with multi-frequency continuum archival VLA data, which also contains the Pie-Town antenna link. This configuration provides even better angular resolution than the VLA alone can produce. I am investigating various imaging techniques and am fitting various models to the calibrated visibilities.\\ 
    
    \item I have recently prepared multi-frequency millimeter observing tracks for a project to use the CARMA interferometer to observe a sample of red giants (PI: E. O'Gorman, ID: C1038). These data have yet to be reduced.\\ 
    
    \item In June 2013, I stayed at the GMRT while our long wavelength radio observations were being carried out (PI: E. O'Gorman, ID: 24$\_$013). I prepared observing scripts and also spent time carrying out initial data analysis at the National Centre for Radio Astrophysics.\\
      
   
   \end{itemize}

\end{document}






