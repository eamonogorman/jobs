% Cover letter using letter.sty
\documentclass[10pt]{letter} % Uses 10pt
%Use \documentstyle[newcent]{letter} for New Century Schoolbook postscript font
% the following commands control the margins:
\topmargin=-0.5in    % Make letterhead start about 1 inch from top of page 
\textheight=8in  % text height can be bigger for a longer letter
\oddsidemargin=0pt % leftmargin is 1 inch
\textwidth=6.5in   % textwidth of 6.5in leaves 1 inch for right margin

\begin{document}

\longindentation=0pt                       % needed to get closing flush left
\let\raggedleft\raggedright                % needed to get date flush left
 
\begin{letter}{NRAO Postdoc \\
Green Bank, WV, USA \\
}

\begin{flushleft}
{\large\bf Eamon O'Gorman}
\end{flushleft}
\medskip\hrule height 1pt
\begin{flushright}
\hfill School of Physics \\
\hfill Trinity College Dublin \\
\hfill Dublin 2 \\
\hfill Ireland \\
\hfill +353 (0)85 720 3415 \\
\hfill eogorma@tcd.ie \\

\end{flushright} 
\vfill % forces letterhead to top of page

\opening{To whom it may concern:} 

 
\noindent I am currently in the final stages of completing a Ph.D. in astrophysics at Trinity College Dublin, under the supervision of Prof. Graham Harper. On September 30, 2013, I submitted my thesis (Title: \textit{Radio Interferometric Studies of Cool Evolved Stellar Winds}), and my thesis defence is due to take place on November 15, 2013. I can assure you that my technical experience, enthusiasm, and scientific knowledge make me a strong candidate to work as an NRAO postdoc at Green Bank. 

\noindent My Ph.D. research has focused on understanding the atmospheres of red giants and red supergiants through multi-wavelength thermal continuum and line emission studies. These stars have substantial mass-loss rates, but the mechanisms which heat and launch their outflows remain an enduring mystery. Progress in this field continues to be driven by observations, and this NRAO postdoc position at Green Bank would enable me to pursue such observations. 

\noindent I have extensive experience in reducing interferometric data at sub-millimeter, millimeter, and centimeter wavelengths, all of which have been carried out within CASA. I have also worked with data at UV and optical wavelengths. I have been a successful P.I. on proposals to use the Giant Metrewave Radio Telescope, the Combined Array for Research in Millimeter-wave Astronomy, and the Karl G. Jansky Very Large Array, and I have designed and planned all such observations. I also have a keen interest in interferometry techniques such as calibration and imaging methods.

Thank you for taking the time to consider my credentials and I look forward to hearing from you.

Eamon O'Gorman
\\
\\
\\
\\
\\
\\
\\
\\
\end{letter}
 

\end{document}






