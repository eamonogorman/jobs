% Cover letter using letter.sty
\documentclass[10pt]{letter} % Uses 10pt
%Use \documentstyle[newcent]{letter} for New Century Schoolbook postscript font
% the following commands control the margins:
\topmargin=-1in    % Make letterhead start about 1 inch from top of page 
\textheight=10.7in  % text height can be bigger for a longer letter
\oddsidemargin=0pt % leftmargin is 1 inch
\textwidth=6.5in   % textwidth of 6.5in leaves 1 inch for right margin

\begin{document}

\longindentation=0pt                       % needed to get closing flush left
\let\raggedleft\raggedright                % needed to get date flush left

\begin{letter}{Postdoctoral Position - Ref 20140309\\
Chalmers University, Sweden
}

\begin{flushleft}
{\large\bf Eamon O'Gorman - Letter of Application}
\end{flushleft}
\medskip\hrule height 1pt
\begin{flushright}
\hfill School of Physics \\
\hfill Trinity College Dublin \\
\hfill Dublin 2 \\
\hfill Ireland \\
\hfill +353 (0)85 720 3415 \\
\hfill eogorma@tcd.ie \\

\end{flushright} 
\vfill % forces letterhead to top of page
\vspace{-1.2cm}
\opening{To whom it may concern:} 

 
\noindent I have recently completed a Ph.D. in astrophysics at Trinity College Dublin, under the supervision of Prof. Graham Harper. My Ph.D. research involved carrying out both millimeter and centimeter interferometric observations of evolved stars, to gain insight into the mechanisms by which they lose mass to the interstellar medium. I am currently working in Trinity College Dublin as a research assistant to Prof. Graham Harper until April 2014, and envisage to continue with my research in stellar astrophysics thereafter. I can assure you that my technical experience in radio interferometric data analysis, scientific knowledge of evolved stellar atmospheres, and communication skills, make me a strong candidate for the postdoctoral research position at Chalmers University.

\noindent My research to date has focused on gaining a broader understanding of the atmospheres of red giants and red supergiants using multi-wavelength radio interferometric observations. These evolved stars have substantial mass-loss rates, but the mechanisms which heat and launch their outflows remain an enduring mystery. Progress in this field continues to be driven by observations, and the latest suite of radio interferometers such as the JVLA, e-MERLIN, and ALMA promise to provide both the sensitivity and spatial resolution required to solve the long standing problem of what drives the substantial mass-loss rates in these stars. As a  postdoctoral researcher at Chalmers University, I would utilize these newly available instruments to improve our understanding of the roles convection, magnetic fields, molecules, and dust play, in transporting mass from the  the optical photosphere to the interstellar medium.

\noindent I have extensive experience in reducing interferometric data at both millimeter and centimeter wavelengths, all of which have been carried out within CASA. During my Ph.D., I have been a successful P.I. on proposals to use the Combined Array for Research in Millimeter-wave Astronomy (CARMA) array, the Karl G. Jansky Very Large Array (VLA), and the Giant Metrewave Radio Telescope (GMRT), and I have designed and planned all of these observations. I thoroughly enjoyed demonstrating and teaching physics to undergraduate students during the course of my Ph.D. and would relish the opportunity to again teach physics as a postdoctoral researcher. I also have a keen interest in public outreach and would like to continue this interest into my postdoctoral career.

\noindent One of my main goals as a postdoctoral researcher is to broaden my research areas in astrophysics and the Radio Astronomy and Astrophysics group at Onsala Space Observatory would provide an ideal opportunity for this. There is a strong overlap between the research areas of the group and my own personal research and so my current research area in evolved stars would complement that of the group. This position would also provide the opportunity to carry out research on even more evolved targets (e.g., post-AGB stars) and possibly even get involved in other areas of research in astrophysics such as star formation. My long term goal is to have a career in astrophysics research and teaching, and this postdoctoral position would act as a stepping stone to achieve that goal.

Thank you for taking the time to consider my credentials and I look forward to hearing from you.

Eamon O'Gorman
\\
\\
\\
\\
\\
\\
\\
\\
\end{letter}
 

\end{document}






