%______________________________________________________________________________________________________________________
% @brief    LaTeX2e Resume for Kamil K Wojcicki
\documentclass[margin,line]{resume}
\usepackage{color}
\usepackage{hyperref}
\usepackage{cleveref}
\hypersetup{citecolor=Blue}
\hypersetup{linkcolor=Blue}


%______________________________________________________________________________________________________________________
\begin{document}
\name{\Large Eamon O'Gorman - CV {\color{white} , \color{black}}}
\begin{resume}

    %__________________________________________________________________________________________________________________
    % Contact Information
    \vspace{2mm}
    \section{\mysidestyle Contact\\Information}

    School of Physics			                            \hfill office: +353 (0)1 896 2157          \vspace{0mm}\\\vspace{0mm}%
    Trinity College Dublin                    				\hfill mobile: +353 (0)85 720 3415  \vspace{0mm}\\\vspace{0mm}%
    Dublin 2											    \hfill e-mail: eogorma@tcd.ie \vspace{0mm}\\\vspace{0mm}%
Ireland					\hfill website: \href{http://www.maths.tcd.ie/~eogorma/index.html}{maths.tcd.ie/{\raise.17ex\hbox{$\scriptstyle\mathtt{\sim}$}}eogorma}
\vspace{0mm}\\\vspace{-4.5mm}%
    %__________________________________________________________________________________________________________________
    % Research Interests
%    \section{\mysidestyle Research\\Interests}

%    Speech processing, speech enhancement, speech and speaker recognition, speech perception, \\ 
%    machine learning and pattern recognition.

    %__________________________________________________________________________________________________________________
    % Honours and Awards
    \vspace{2mm}
    \section{\mysidestyle Research\\Statement} 
My research to date has focused on stellar evolution, with a specific emphasis on late K and early M spectral type red giants and red supergiants. These stars have substantial mass-loss rates yet the mechanisms which drives this large mass-loss are unknown and remain one of the great unsolved problems in modern stellar astrophysics. To gain insight into the mass-loss mechanisms, I have observed these stars using both millimeter and centimeter radio interferometric techniques, which have probed both their wind acceleration zones and their circumstellar environments. As a post-doctoral researcher at the ASIAA I plan to continue such millimeter and centimeter interferometric observations using instruments such as ALMA, the JVLA, and e-MERLIN. I believe my research interests complement those of the  interstellar and circumstellar (ICSM) group at the ASIAA, while my experience in radio interferometric data analysis would also be of great benefit to the ASIAA group in general.


    %__________________________________________________________________________________________________________________
    % Education
    \vspace{2mm}
    \section{\mysidestyle Education}

    \textbf{Trinity College Dublin}, Dublin, Ireland \vspace{1mm}\\\vspace{0.5mm}%
    \textsl{Doctor of Philosophy} \hfill \textbf{ October 2009 -- present}\vspace{-3mm}\\\vspace{-1mm}%
    \begin{list2}
        \item Advisor:  Professor Graham Harper
        \item  Thesis title: \textit{Radio Interferometric Studies of Cool Evolved Stars}
        \item  Thesis successfully defended on 15th November 2013
            \begin{list2}
        	\item External Examiner:  Professor Tom Millar (Queen's University Belfast, Northern Ireland)
        	\item Internal Examiner:  Professor Peter Gallagher
        	\end{list2}\vspace{-1.0mm}
    \end{list2}\vspace{-1.0mm}
    \textbf{International Space University (ISU)}, Strasbourg, France \vspace{0.5mm}\\\vspace{1mm}%
     M.Sc., Space Studies \hfill \textbf{ September 2008 -- August 2009}\vspace{3mm}\\\vspace{-1mm}
   \textbf{\hspace{-2.3mm} University College Dublin (UCD)}, Dublin, Ireland \vspace{1.5mm}\\\vspace{1mm}%
     B.Sc., Theoretical Physics, (First class honours) \hfill \textbf{ September 2003 -- June 2007}\vspace{-2mm}\\\vspace{-1mm}%

    %__________________________________________________________________________________________________________________
    % Honours and Awards
    \vspace{2mm}
    \section{\mysidestyle Honours and\\Awards} 
	\begin{list2}
    \item Enterprise Ireland/European Space Agency scholarship to study at the ISU, 2008
    \item UCD Entrance Scholar, 2003 \vspace{-1mm}\\%
    \end{list2}   

    %__________________________________________________________________________________________________________________
    % Publications
    \section{\mysidestyle Refereed Publications}

    \textbf{O'Gorman, E.}, Harper, G. M., Brown, A., Drake, S., Richards, A. M. S.
    \textsl{Multi-wavelength Radio Continuum Emission Studies of Dust-free Red Giants},
    2013, AJ, 146, 98.

\vspace{0mm}
    Richards, A. M. S., Davis, R. J., Decin, L., Etoka, S., Harper, G. M., Lim, J. J., Garrington, S. T., Gray, 	M. D., McDonald, I., \textbf{O'Gorman, E.}, Wittkowski, M.
    \textsl{e-MERLIN resolves Betelgeuse at $\lambda$ 5\,cm: hotspots at $5\,R_{\star}$},
    2013, MNRAS, 432, L61.

\vspace{0mm}
	\textbf{O'Gorman, E.}, Harper, G. M., Brown, J. M., Brown, A., Redfield, S., Richter, M. J., Requena-			Torres, M. A.
    \textsl{CARMA CO($J = 2-1$) Observations of the Circumstellar Envelope of Betelgeuse},
    2012, AJ, 144, 36.

\vspace{0mm}
	Sada, P. V., Deming, D., Jennings, D. E., Jackson, B. K., Hamilton, C. M., Fraine, J., Peterson, S. W., 		Haase, F., Bays, K., Lunsford, A., \textbf{O'Gorman, E.}
    \textsl{Extrasolar Planet Transits Observed at Kitt Peak National Observatory},
    2012, PASP, 124, 212.

\vspace{0mm}
	Sada, P. V., Deming, D., Jackson, B., Jennings, D. E., Peterson, S. W., Haase, F., Bays, K., \textbf{O'Gorman, E.}, Lundsford, A.
    \textsl{Recent Transits of the Super-Earth Exoplanet GJ 1214b},
    2010, ApJ, 720, L215.



    %__________________________________________________________________________________________________________________
    % Professional Experience
    \vspace{2mm}
    \section{\mysidestyle Conference\\Presentations}

    \textbf{Oral:} \vspace{2mm}\\\vspace{1mm}%
\textbf{O'Gorman, E.} \textit{Radio Interferometric Studies of Cool Evolved Stellar Mass Outflows.}
DIAS Seminar, Dublin, Dublin Institute for Advanced Studies, Ireland, February 2013. %    
    
	\textbf{O'Gorman, E.}, et al. \textit{Probing the Thermodynamics of Red Giant Mass Outflows with the JVLA.}
Astronomical Science Group of Ireland, Galway, Ireland, November 2012. %    
    
\textbf{O'Gorman, E.}, et al. \textit{Probing the Thermodynamics of Red Giant Mass Outflows with the JVLA.}
Radio Stars and Their Lives in the Galaxy, MIT Haystack Observatory, MA, USA, October 2012. %

\textbf{O'Gorman, E.} \textit{CO in the Circumstellar Envelope of Betelgeuse with CARMA.}
The 41st Young European Radio Astronomers Conference, University of Manchester/Jodrell Bank Observatory, UK, July 2011.

    \textbf{Poster:} \vspace{2mm}\\\vspace{1mm}%
    \textbf{O'Gorman, E.}, Harper, G. M. \textit{What is Heating Arcturus' Wind?}\\
16th Cambridge Workshop on Cool Stars, Stellar Systems, and the Sun, University of Washington, Seattle, USA, August 2010.


    %__________________________________________________________________________________________________________________
    % Proposals
    \vspace{2mm}
    \section{\mysidestyle Accepted\\Proposals\\as PI}

\textbf{O'Gorman, E.}, et al. \textit{Beta Gem b: An alternative candidate in the search for extrasolar planetary radio emission (II)}, GMRT, 2013, ID: 25\textunderscore 039%

\textbf{O'Gorman, E.}, et al. \textit{Beta Gem b: An alternative candidate in the search for extrasolar planetary radio emission}, GMRT, 2013, ID: 24\textunderscore 013%

\textbf{O'Gorman, E.}, et al. \textit{Thermal Continuum Mapping of Red Giant Chromospheres}, CARMA, 2012, ID: c1038%

\textbf{O'Gorman, E.}, et al. \textit{L and S band Continuum Observations of Arcturus: Completing a Clean Sweep}, VLA, 2012, ID: VLA-12A-472%

    %__________________________________________________________________________________________________________________
    % Proposals
    \vspace{2mm}
    \section{\mysidestyle Short-term\\Research\\Stays}

\textbf{National Centre for Radio Astrophysics, India, 2013:} Collaboration with Dr. Sandeep Sirothia to prepare our GMRT 150\,MHz observations and carry out initial data analysis.

\textbf{Harvard-Smithsonian Center for Astrophysics, USA, 2011:} Collaboration with Dr. Joanna Brown to carry out initial analysis on our CARMA millimeter data using CASA. 

\textbf{NASA Goddard Space Flight Center, USA, 2009:} Three month student internship with Dr. Drake Deming in the area of transiting exoplanet characterization. Analyzed data from the FLAMINGOS infrared camera on the 2.1\,m Kitt Peak National Observatory Telescope.

    %__________________________________________________________________________________________________________________
    % Proposals
    \vspace{2mm}
    \section{\mysidestyle Teaching\\Experience}

	\begin{list2}
    \item September 2010 - April 2012: Bi-weekly physics tutorials for undergraduate engineering students.
    \item September 2009 - April 2010: Teaching assistant for undergraduate physics students. \vspace{-1mm}\\%
    \end{list2}

    %__________________________________________________________________________________________________________________
    % Proposals
    \vspace{0mm}
    \section{\mysidestyle Outreach}

	\begin{list2}
    \item Throughout the year we carry out a ``Build a Comet in the Lab'' workshop for both secondary school students interested in pursuing physics in college, and for primary school students from disadvantaged backgrounds.
    \item Regularly visit secondary schools to discuss career opportunities in physics and astrophysics to prospective students.
        \item Active member in ``SunSpotter'', a new project which aims to enlist the help of the public to readily identify and characterize sunspots in NASA satellite image.\vspace{-1mm}\\%
    \end{list2}


    %__________________________________________________________________________________________________________________
    % Computer Skills
    \vspace{0mm}
    \section{\mysidestyle Programming} 

    IDL, Python, CASA, \LaTeXe, PHP, BASH, and UNIX--like operating systems.

    % Computer Skills
    \vspace{0mm}
    \section{\mysidestyle Languages} 
	\begin{list2}
    \item English (Native)
    \item Good Irish and French
    \end{list2}


    %__________________________________________________________________________________________________________________
    % Computer Skills
    \vspace{0mm}
    \section{\mysidestyle Professional\\Organisations} 
	\begin{list2}
    \item Fellow of the Royal Astronomical Society
    \item Associate Member of the Institute of Physics \vspace{-1mm}\\%
    \end{list2}
    
    %__________________________________________________________________________________________________________________
    % Referees
%    \section{\mysidestyle Referees} 
%    {\sl Available on request.}

%______________________________________________________________________________________________________________________
\vspace{0mm}
\section{\mysidestyle Referees} 

\begin{tabular}{@{}p{4.6cm}p{4.6cm}p{4.8cm}}
\textbf{Prof. Graham Harper}       &  \textbf{Prof. Peter Gallagher}   &  \textbf{Prof. Hugh Hill} \\
School of Physics                      &  School of Physics                   &  International Space University \\
Trinity College Dublin                 &  Trinity College Dublin                   &  1 rue Jean--Dominique Cassini \\
Dublin 2, Ireland				       &  Dublin 2, Ireland		         &  Strasbourg, France \\
phone: \textsl{+353 (0)1 896 3257}    &  phone: \textsl{+353 (0)1 896 1300} &  phone: \textsl{+33 (0)3 88 65 54 39} \\
e-mail: \textsl{graham.harper@tcd.ie}   &  e-mail: \textsl{peter.gallagher@tcd.ie}&  e-mail: \textsl{hill@isu.isunet.edu} \\
\end{tabular}




%______________________________________________________________________________________________________________________
\end{resume}
\end{document}


%______________________________________________________________________________________________________________________
% EOF
