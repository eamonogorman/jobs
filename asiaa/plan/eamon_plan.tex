% Cover letter using letter.sty

\documentclass[11pt]{letter} % Uses 10pt
\usepackage{graphicx,subfigure}
\usepackage{float}
\usepackage{caption}
\captionsetup{labelsep=space}

\renewcommand*{\figureformat}{\figurename~\thefigure}

\restylefloat{figure}
%Use \documentstyle[newcent]{letter} for New Century Schoolbook postscript font
% the following commands control the margins:
\topmargin=-1.2in    % Make letterhead start about 1 inch from top of page 
\textheight=14in  % text height can be bigger for a longer letter
\oddsidemargin=0pt % leftmargin is 1 inch
\textwidth=6.5in   % textwidth of 6.5in leaves 1 inch for right margin
\textheight 255mm


\begin{document}\pagenumbering{gobble}




\begin{flushleft}
{\large\bf Eamon O'Gorman}
\end{flushleft}
\medskip\hrule height 1pt
\begin{flushright}


\end{flushright} 
%\vfill % forces letterhead to top of page
{\Large 
\begin{center}
Summary of Past Research
\end{center}
}

My main research achievements to date have been associated with gaining a broader understanding into the inner atmospheres and circumstellar environments of red giants (non-AGB red giants) and red supergiants (RSGs), through radio interferometric techniques. The mass-loss from these stars plays a crucial role in galactic evolution and ultimately provides much of the material required for the next generation of stars and planets. Also, a substantial fraction of the star's initial mass can be dispersed to the interstellar medium during these post main sequence evolutionary stages and this mass-loss is therefore a crucial factor governing stellar evolution, and also in explaining the frequency of supernovae in the galaxy. Despite the importance of this phenomenon, and decades of study, the specific mechanisms that drive winds from evolved spectral-type K through mid-M stars remain unknown. There is insufficient atomic, molecular, or dust opacity to drive a radiation-driven outflow and acoustic/pulsation models cannot drive the observed mass-loss rates. UV and optical observations reveal an absence of significant hot wind plasma, and the winds are thus too cool to be Parker-type thermally-driven flows. Magnetic fields are most likely involved in the mass-loss process, although current magnetic models are also unable to explain spectral diagnostics. Traditionally, observations have provided only limited disk-averaged information about the outflow environments of these stars, making it difficult to infer the wind properties. My primary research to date has focused on using the latest suite of millimeter and centimeter interferometers to provide essential spatial information on these outflow environments, to gain a better understanding of the entire mass-loss process.

Part of my research to date has focused on understanding the complex circumstellar environment of the enigmatic M supergiant Betelgeuse. Betelgeuse is one of the few nearby RSGs that can be studied in great detail across most of the electromagnetic spectrum. The extended atmosphere of this oxygen-rich RSG provides an ideal testbed for studying the poorly understood processes that drive mass-loss from K and M supergiants. Betelgeuse has undergone at least two distinct phases of mass-loss in its recent past. However, previous single dish millimeter observations were only able to detect one component of the outflow, while the spatial extent of both components were entirely unknown. Using multiple array configurations of the CARMA millimeter interferometer, we successfully imaged the CO($J = 2 - 1$) emission line at sub-arcsecond resolution, and for the first time were able to find the spatial extent of both outflow components, allowing their ages to be calculated (O'Gorman et al., 2012, \textit{AJ}). We found both flow components to be inhomogeneous, far from the existing spherically symmetric models of its circumstellar environment, indicating a chaotic mass-loss loss process. We also observed  the CO($J = 3 - 2$) emission line with the SMA millimeter interferometer and we found that both emission lines probed similar structure in the circumstellar environment (O'Gorman, 2013, \textit{Thesis}). We also analyzed higher rotation CO lines using the Herschel and the SOFIA observatories to calculate the excitation temperature of the inner flow of Betelgeuse (O'Gorman, 2013, \textit{Thesis}). The millimeter interferometric line emission data were fully reduced by myself using CASA. The analysis required knowledge of interferometric spatial scales and imaging techniques such as CLEAN and multi-scale CLEAN. 

The wind acceleration region is where most of the momentum and heat which drive the mass-loss in RSGs is deposited. Thermal millimeter and centimeter continuum emission directly sample this region which extends out to only a few $R_{\star}$. I have been heavily involved in a collaboration which has used e-MERLIN to image the thermal continuum emission from this inner region of Betelgeuse's atmosphere at 6\,cm (Richards et al., 2013). We have discovered two chromospheric ``hotspots'' (Figure 1, \textit{left}) separated by $4\,R_{\star}$, hidden just beyond the spatial resolution of the VLA at 6\,cm. Using the astrometric solution of Harper et al. (2008), the brighter hotspot ($T_{e} > 5400$\,K) is $\sim 3.5\,R_{\star}$ from the photosphere. Inspired by this new discovery, I have analyzed multi-wavelength, multi-epoch, centimeter VLA data which contains the Pie Town VLBA antenna, to look for signatures of these hotspots. At the highest available frequencies, these data have even superior spatial resolution than e-MERLIN (at 6\,cm).  In Figure 1 (right) a preliminary map at 0.7\,cm is presented, and exciting evidence of at least two features separated by only $2\,R_{\star}$ is shown. These features may be generated by  giant photospheric convective cells, although it is unlikely that such features could be accountable for the ``hotspots'' seen with e-MERLIN (O'Gorman, 2014, \textit{In prep}).

We have recently used the Jansky VLA to observe two non-dusty, non-pulsating, K spectral-type red giants at multiple radio wavelengths ($0.7 - 20$\,cm) (O'Gorman et al., 2013, \textit{AJ}). Such stars are feeble emitters at these wavelengths however, and previous observations have provided only a small number of modest S/N measurements slowly accumulated over three decades. Our observations of each star were carried out over just a few days, so that we obtained an essentially consistent snapshot of the different stellar atmospheric layers sampled at different wavelengths. We found that our observations were in disagreement with the existing semi-empirical atmospheric models for these stars, which were based only on UV diagnostics. We also found evidence for a rapidly cooling stellar wind for one of the targets which allowed us to develop a new semi-empirical wind model for the star. This model was then used as the basis to compute a thermal energy balance of the star's outflow by investigating the various heating and cooling processes that control its thermal structure (O'Gorman, 2013, \textit{Thesis}). As part of this project, I planned and prepared two sets of observations of multi-frequency centimeter continuum observations with the Jansky VLA. Different frequencies required different observing strategies, such as choice of calibrator (both flux and phase) and length of observing scans. All data analysis was again carried out in CASA. The large bandwidth now provided by the Jansky VLA required additional steps to mitigate RFI, and to create wideband images.

\begin{figure}[!ht]
\centering 
\mbox{
          \includegraphics[trim=27pt 90pt 40pt 90pt,clip,angle=90,width=8cm,height=7cm]{/home/eamon/jobs/naasc/plan/fig1.ps}
          \includegraphics[trim=145pt 587pt 140pt 20pt,clip,width=9.5cm,height=7.0cm]{/home/eamon/jobs/naasc/plan/fig0.ps}
          }
\caption{ {\small  \textbf{Figure 1.} \textit{Left:} e-MERLIN 6\,cm image of Betelgeuse showing the two chromospheric ``hotspots''. The red filled circle marks the expected photospheric position. \textit{Right:} VLA + Pie Town 0.7\,cm image of Betelgeuse's inner asymmetric atmosphere. The presence of giant photospheric convective cells can account for the asymmetries in the 0.7\,cm image, but cannot account for the hotspot at $\sim 3.5\,R_{\star}$ in the 6\,cm image.}}
\end{figure}

{\Large 
\begin{center}
Future Research Plan
\end{center}
}
The latest suite of radio interferometers such as ALMA, the Jansky VLA, and e-MERLIN now have the capability of providing spatially resolved sensitive millimeter and centimeter observations of the atmospheres of the closest RSGs and red giants. Along with Taiwan being a member country of the ALMA project, a postdoctoral position at the ASIAA would provide me with both the time and resources to carry out my future millimeter and centimeter observations. Such observations would contribute exciting new insights into the currently unknown mass-loss process in RSGs and red giants. 
\begin{center}
\textbf{Red Supergiants}
\end{center}
I am part of a collaboration that has been awarded ALMA cycle 1 time which will trace CO($J=6-5$) emission around Betelgeuse at a resolution 10 times higher (i.e., $\sim 0.1''$) than we achieved with the CARMA interferometer (discussed previously). Such observations will give a detailed picture into the dynamics of the inner atmosphere of M supergiants. There has also been much debate into what role dust plays in the mass-loss from RSGs and these observations will also be capable of imaging any dust which may lie just beyond $\sim 2.5\,R_{\star}$ of Betelgeuse, a relatively non-dusty M supergiant. These observations may be carried over to Cycle 2 if our project is not fully observed during Cycle 1 (i.e., by the end of May 2014). My past experience with analyzing millimeter interferometric data will be of great importance to this project and working with the first ALMA observations of a RSG may be my first project as a postdoctoral researcher at the ASIAA. In year 2 as a postdoctorate at the ASIAA, ALMA should have sufficient spatial resolution ($\sim 50$ mas) at multiple frequencies to spatially resolve the free-free thermal emission from Belelgeuse and Antares (the two closest M supergiants). I envisage using ALMA at multiple frequencies to directly determine the temperature profiles of the  innermost atmospheres ($<2\,R_{\star}$) of these stars. Such measurements have previously been carried out at longer wavelengths with the VLA and have probed the temperature between $2 < \,R_{\star} < 5$. However, observations at ALMA frequencies will probe the innermost atmosphere where much of the energy that goes into driving the wind is deposited, and will confront existing models along with searching for evidence of the giant convection cells and \textit{chromospheric hotspots} seen at longer wavelengths (i.e., with the VLA and e-MERLIN respectively).

Our recent findings with the very long baseline interferometer, e-MERLIN, raise more questions about the mass-loss process in M supergiants than answers. What mass-loss process could cause such chromospheric hotspots? On what time scales do they evolve? Multi-wavelength high spatial resolution monitoring of Betelgeuse with e-MERLIN is required to solve this puzzling evidence and this project would be one of my main priorities in my first year at the ASIAA. The next call for e-MERLIN proposals is in spring 2014 and I plan to submit a strong proposal to observe Betelgeuse at both 6\,cm (C band) and 1.3\,cm (K band) for at least two epochs. This multi-epoch multi-wavelength data will provide insights into the time scales on which the chromospheric hotspots evolve and may also be capable of spatially resolving them allowing their size and temperature to be directly determined. I also plan to apply for Jansky VLA A-configuration time in August 2014 to observe Betelgeuse again at 6\,cm and 1.3\,cm with the intention of combining these data with data from e-MERLIN. Such data would produce the highest spatial resolution radio image ever of the thermal emission from any star other than our Sun. 
\begin{center}
\textbf{Red Giants}
\end{center}
The Jansky VLA now provides over an order of magnitude increase in continuum sensitivity. Another goal in my first year at the ASIAA would be to utilize this new capability and carry out a survey ($\sim 10 - 15$ targets) of nearby coronal red giants to detect their total radio flux density at 3.5\,cm and 6\,cm. Such measurements will provide estimates of their mass-loss rates, which are notoriously difficult to estimate at other wavelengths. Plotting mass-loss rates as a function of various stellar parameters will for the first time allow empirical mass-loss formulae to be constructed for these late-type stars. To avoid confusion from near-by extragalactic objects, the second most extended B-configuration of the Jansky VLA will be needed and so I again plan on proposing for time with the Jansky VLA in August 2014.

The atmospheres of red giants cannot currently be spatially resolved at millimeter or centimeter wavelengths, but this will change in the next few years when ALMA and e-MERLIN achieve sufficient spatial resolution. The atmospheric properties of these stars are still poorly understood and this ultimately leads to a lack of understanding into the mechanism by which they lose mass. As well as revealing any large scale structure that may be present in their wind acceleration zones, spatially resolved thermal free-free millimeter and centimeter emission can directly provide a measurement of the gas temperature. It is the gas temperature which controls the heating rates in the wind, which in turn provides insights into the unknown mass-loss mechanism. I envisage to carry out such studies in my second year at the ASIAA.
\begin{center}
\textbf{Other Interests}
\end{center}
My interests in astrophysics are wide and varied and I would also be very much interested in starting new collaborations at the ASIAA and indeed at NRAO where I could utilize my experience in millimeter and centimeter interferometry in other areas of astrophysics. An example of this is my ongoing project to detect radio emission from $\beta$ Gem b, an exoplanet orbiting the closet red giant, Pollux (PI: E. O’Gorman, ID: 24 013). Previous searches for exoplanet radio emission have focused on close orbiting planets, the so-called \textit{hot Jupiters}. Our target is further away from its host star and free from tidal locking, which may reduce the internal magnetic field of the hot Jupiters, thus reducing the strength of the radio emission. We expect the relatively large mass-loss of the host red giant to be a key driver in detecting this emission. We are currently using the GMRT to search for this emission at 150\,MHz. In June 2013, I stayed at the GMRT while our long wavelength radio observations were being carried out. While there I prepared observing scripts and carried out initial data analysis. At the ASIAA, I would envisage to carry out further observations of this interesting planetary system at lower frequencies and higher sensitivity with LOFAR. 

\end{flushright}
\end{document}






